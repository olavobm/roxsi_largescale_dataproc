%\documentclass[a4paper,11pt,oneside]{report}  % Draft settings
\documentclass[a4paper,11pt]{book}  % 11pt in final form

\newcommand{\FileId}{File: home\_exercises.tex, Last changed: 2005-04-29}
\newcommand{\IsDraft}{0} % 0 = Final version, 1 = Draft version

% Packages
\usepackage[T1]{fontenc}
\usepackage{graphics}
\usepackage{epsfig}
\usepackage{latexsym}
\usepackage{rotating}
\usepackage{fancyheadings}
\usepackage{amssymb}
\usepackage{ifthen}
\usepackage{verbatim}

% Local pacages
\usepackage{litenfig}

% Style-file for LAB
\usepackage{lab}


%%%%%%%%%%%%%%%%%%%%%%%%%%%%%%%%%%%%%%%%%%%%%%%
% Set Font for LAB
%%%%%%%%%%%%%%%%%%%%%%%%%%%%%%%%%%%%%%%%%%%%%%%

%\usepackage{times}             % Times
\usepackage{palatino}          % Palatino
%\renewcommand{\rmdefault}{pad}  % Garamond
%\renewcommand{\rmdefault}{pfr}  % Frutiger

\begin{document}

%%%%%%%%%%%%%%%%%%%%%%%%%%%%%%%%%%%%%%%%%%%%%%%
% Title page
%%%%%%%%%%%%%%%%%%%%%%%%%%%%%%%%%%%%%%%%%%%%%%%

%\thispagestyle{empty}
\pagestyle{plain}

\begin{center}
  {\bf {\Huge Random Loads with Data Analysis\\[2mm]
               {\LARGE --- Home Exercises ---}}
  \ifthenelse{\equal{\IsDraft}{1}}{%
    \footnotetext{\FileId \hfill Compiled: \today}}{}

\vskip1cm

\begin{tabular}[t]{c@{\qquad}c}
\emph{P�r Johannesson}      & \emph{Igor Rychlik} \\[.5mm]
Fraunhofer-Chalmers Centre  & Mathematical Statistics \\
                            & Chalmers University of Technology \\[.5mm]
\end{tabular}

\vskip5mm

\ifthenelse{\equal{\IsDraft}{1}}{Draft: 2005-04-14}{Version
2005-04-29} }
\end{center}


\vskip5mm

%%%%%%%%%%%%%%%%%%%%%%%%%%%%%%%%%%%%%%%%%%%%%%%
% Home Exercises
%%%%%%%%%%%%%%%%%%%%%%%%%%%%%%%%%%%%%%%%%%%%%%%

\section*{Part A - Solve using Paper \& Pen}

\begin{description}
    \item[A.1] Let $X(t)$ be a stationary Gaussian random process with zero mean and spectral
    density
    $$S(\omega)=C/\omega^k, \quad \omega_1<\omega<\omega_2 , \qquad \mbox{with~~~$k=5$} .$$
    This type of spectrum is often used as models for roads.
    \begin{itemize}
        \item Compute the level upcrossing intensity, $\mu(u)$.
        \item How can you use $\mu(u)$ to compute an upper bound for
        the rainflow damage?  How is this connected to the
        narrow band approximation?  Derive the corresponding distribution
        of rainflow cycle amplitudes!
        \item How can we construct a lower bound for the rainflow
        damage?
        \item A simple approximation of the min-max distribution $f_{m,M}(u,v)$ is
        obtained by using
        $$M=\sigma(\alpha_2 R + \sqrt{1-\alpha_2^2}X), \qquad \mbox{and} \qquad
          m=\sigma(-\alpha_2 R + \sqrt{1-\alpha_2^2}X)$$
        where $R$ and $X$ are standard
        Rayleigh and Normal distributions, respectively.  Use this
        to derive an approximate result for the lower bound for the
        rainflow damage.
        \item Use the extreme value approximation
        $$\mu_{\rfc}(u,v)=\frac{\mu(u)\mu(v)}{\mu(u)+\mu(v)}$$
        to compute an approximation of the rainflow damage.
    \end{itemize}
    P.S. For the damage calculations you can assume the Basquin
    equation $N=\alpha S^{-\beta}$ with $\beta=4$.
\end{description}

\newpage

\section*{Part B - Solve using Matlab \& WAFO}

\begin{description}
    \item[B.1] Use the Gaussian process in A.1 with $C=1$, $\omega_1=1$, and $\omega_2=10$.
    \begin{itemize}
        \item Compute the limiting rainflow matrix and the damage.
        \item Compute the upper and lower bounds for the rainflow
        damage.
        \item Compare the exact rainflow damage with the upper
        and lower bound, as well as with the extreme value
        approximation.
        \item Simulate a sample path of the Gaussian process and compute the damage.
        How does it compare to the other computations of the damage?
    \end{itemize}
    \item[B.2] Here we will analyse a measured signal.  You can
    choose a signal of your own, or use \verb+sea.dat+ in WAFO.
    \begin{itemize}
        \item Compute the rainflow matrix and the Markov matrix from
        the signal.
        \item Now suppose that we have lost the original signal, as
        well as the rainflow matrix, and only have the Markov
        matrix.  How can we then compute approximations or bounds
        for the rainflow damage?  Use at least three different
        methods/approximations.
    \end{itemize}
\end{description}

\end{document}
