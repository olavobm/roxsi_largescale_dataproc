\renewcommand{\FileId}{File: lab\_intro.tex, Last changed: 2005-04-16}

%%%%%%%%%%%%%%%%%%%%%%%%%%%%%%%%%%%%%%%%%%%%%%%
% Introduction
%%%%%%%%%%%%%%%%%%%%%%%%%%%%%%%%%%%%%%%%%%%%%%%

%
% Introduction.
%
\section*{General Introduction}
\markboth{General Introduction}{General Introduction}

This course is intended to present some tools for analysis of (random) loads
in order to assess the fatigue damage.
Throughout the course we will use the Matlab toolbox WAFO
(Wave Analysis for Fatigue and Oceanography). We shall assume that
the load is given by one of three possible forms:

\begin{enumerate}
\item As measurements of the stress or strain function with some given
sampling frequency in Hz. Such loads will be called measured loads
and denoted by $x(t)$, $0\le t\le T$, where $t$ is time and $T$ is the
duration of the measurements.
\item In the frequency domain (that is important in system analysis)
as a power spectrum. This means that the signal is represented by a
Fourier series
\begin{displaymath}
x(t)\approx
m + \sum_{i=1}^N a_i\cos(\omega_i\,t)+b_i
\sin(\omega_i\,t)
\end{displaymath}
where $\omega_i=i\cdot 2\pi/T$ are angular
frequencies,
$m$ is the mean of the signal and $a_i,b_i$ are Fourier coefficients.
\item In the rainflow domain, i.e.\ the measured load is given in the
form of a rainflow matrix.
\end{enumerate}

We shall now review some simple means of characterizing
and analysing loads that are given in forms 1)--3), and how to derive some
characteristics, important for fatigue evaluation and testing.
More details will also be given in exercises.

We assume that the reader has some knowledge about the concept of
cycle counting, in particular rainflow
cycles, and damage accumulation using Palmgren-Miners linear damage
accumulation hypotheses. The basic definitions are given in the end of
this introduction.


\subsection*{Parameters for Measured Load Histories}

Some general properties of measured loads can be
summarized by using a few simple characteristics. Those are
\emph{the mean} $m$, defined as the average of all values, which is
approximately equal to $m=1/T\,\int_0^T x(t)\,dt$, and \emph{the
variance} $\sigma^2$ that measures the variability around the mean
and is defined as $\sigma^2=1/T\,\int_0^T (x(t)-m)^2\,dt$,
\emph{the mean frequency} $f_0$ defined as the number of times $x(t)$
crosses upwards (upcrosses) the mean $m$ normalized by the length
of the observation interval $T$, and \emph{the irregularity factor}
$\alpha$, defined as the intensity of mean upcrossings $f_0$
divided by the intensity of local maxima (intensity of cycles)
in $x(t)$. (Note, a small $\alpha$ means an irregular process,
$(0 < \alpha \leq 1$).)
Another important property
is the crossing spectrum $\mu(u)$ defined as the intensity of
upcrossings of a level $u$ by $x(t)$ as a function of $u$.
Obviously $f_0=\mu(m)$.

The process of damage accumulation depends only on the values
and the order of the local extremes in the load. The sequence of local
extremes is called the \emph{sequence of turning points}. The irregularity
factor $\alpha$ is measuring how dense the local extremes are
relatively to the mean frequency $f_0$. For a regular function
it would be only one local maximum between upcrossings of the mean level
giving irregularity factor equal to one. In the other extreme case,
there are infinitely many local extremes giving irregularity factor zero.
However, if the crossing intensity $\mu(u)$ is finite, most of those
local extremes are irrelevant for the fatigue and should be
disregarded.
A particularly useful filter is the so-called rainflow filter that
removes all local extremes that builds rainflow cycles with amplitude
smaller than a given threshold. We shall always assume that the signals
are rainflow filtered.


\subsection*{Fatigue Life Prediction}
Obviously when the signal is given, the rainflow cycles can be extracted
and  fatigue damage analysis performed. However, often the observed function
is too short to contain all possible cycles that a structure can
experience and there is a need to model the damage when $x(t)$ is
modelled as a possible outcome of a ``random'' measurement or,
more precisely, as a random process, denoted in the following by $X(t)$.
The main objective is then to predict the fatigue life from the
specification of a random load $X(t)$. This problem is resolved on several
levels of complexity.

First we shall use the fact that the crossing intensity
can be used to give a conservative estimate (overestimation) of the accumulated damage caused by $X(t)$, see
Rychlik~\cite{Rychlik93.A2} for algorithm and more detailed discussion.
Now the crossing intensity can be computed using the so-called Rice's
formula. Another possibility is to include the intensity in the model
specification as is done for the so-called transformed Gaussian
loads.

If more accurate predictions of fatigue life are needed then
more detailed models are required for the sequence of turning points.
Here the Markov chain theory has shown to be particularly useful.
There are two reasons for this:
\begin{itemize}
\item the Markov models constitute a broad
class of processes that can accurately model many real loads
\item for Markov models, the fatigue damage prediction using rainflow method
is particularly simple, Rychlik~\cite{Rychlik88.A1} and
Johannesson~\cite{Johannesson99.PhD}
\end{itemize}
In the simplest case, the necessary
information is the intensity of pairs of local maxima and the following
minima (the so-called Markov matrix or min-max matrix). The dependence
between other extremes is modelled using Markov chains,
see Frendahl~\& Rychlik~\cite{Frendahl93.A1}.

%\subsection*{Fatigue Analysis of Measured Load Histories}
%
%[PJ writes.]

%\subsection*{Upper and Lower Bound for Rainflow Damage}
%
%[IR writes.]


\subsection*{Frequency Modelling of Load Histories}

The important characteristic of signals in frequency domain is their
power spectrum  $\hat{s}_i=(a_i^2+b_i^2)/(2\Delta\omega)$, where $\Delta\omega$
is the sampling interval in frequency domain, i.e. $\omega_i=i\cdot \Delta\omega$.
The two-column matrix $\hat{s}(\omega_i)=(\omega_i,\hat{s}_i)$ will be called the power
spectrum of $x(t)$.

The sequence $\theta_i=\arccos(a_i/\sqrt{2\hat{s}_i\Delta\omega})$
is called a sequence of
phases and the Fourier series can be written as follows
$$
x(t)\approx m + \sum_{i=1}^N \sqrt{2\hat{s}_i\Delta\omega}
\cos(\omega_i\,t+\theta_i).
$$
If the sampled signal contains
exactly $2N+1$ points then $x(t)$ is equal to its Fourier series at the
sampled points. In the special case when $N=2^k$, the so-called FFT
(Fast Fourier Transform) can be used in order to compute the Fourier
coefficients (and the spectrum) from the measured signal and in
reverse the signal from Fourier coefficients.

As we have written before, the Fourier coefficient to the zero frequency
is just the mean of the signal, while the variance is given by
$\sigma^2=\Delta\omega\sum \hat{s}(\omega_i)
\approx \int_0^\infty \hat{s}(\omega)\,d\omega$. The last integral
is called the zero-order spectral moment $\lambda_0$. Similarly
higher-order spectral moments are defined by
$$
\lambda_i=\int_0^\infty \omega^i\hat{s}(\omega)\,d\omega.
$$
%In oceanography the satellite measurements give information
%on the  variability of the sea surface by
%giving the two spectral moments $\lambda_0$,  $\lambda_2$ and the main
%frequency $f_0$. Geographical location plus these three parameters are
%then used to select and appropriate energy spectrum $S(\omega)$.

\subsubsection{Random Functions in Spectral Domain}
Assume that we get new measurements of a signal
that one is willing to consider as equivalent, but it is seldom
identical to the first one. Obviously it will have a
different spectrum $\hat{s}(\omega)$ and the phases will be changed.
A useful mathematical model for such a situation are the so-called
random functions (stochastic processes) which will be denoted by $X(t)$.
Here $x(t)$ is seen as particular randomly chosen function.
The simplest case that models
stationary signals with a fixed spectrum $\hat{s}(\omega)$ is
$$
X(t)= m + \sum_{i=1}^N \sqrt{\hat{s}_i\Delta\omega}
\sqrt{2}\cos(\omega_i\,t+\Theta_i),
$$
where $\Theta_i$ are independent uniformly distributed phases.
However, it is not a very realistic model, since in practice we often
observe variability in spectrum $\hat{s}(\omega)$ between measured functions
and hence $\hat{s}_i$ should be modelled as random variables too.
Here we assume that
there is a deterministic function $S(\omega)$ such that the average value
of $\hat{s}(\omega_i)\Delta\omega$ can be approximated by
$S(\omega_i)\Delta\omega$ and in many cases one can model
$\hat{s}_i=R_i^2\cdot S(\omega_i)/2$ where $R_i$ are independent random
factors,
all Rayleigh distributed. (Observe that the average value of $R_i^2$
is 2.)  This gives the following random function
$$
X(t)= m + \sum_{i=1}^N \sqrt{S(\omega_i)\Delta\omega}
R_i\cos(\omega_i\,t+\Theta_i).
$$
The process $X(t)$ has many useful properties that can be used in
analysis like: for any fixed $t$, $X(t)$ is normally distributed, called also
Gaussian distributed. A probability of any event defined for $X(t)$
can, in principal, be computed when the mean $m$ and the spectral
density $S$ are known.

If $Y(t)$ is an output of a linear filter
with $X(t)$ on the input, then  $Y(t)$ is also normally distributed
and we need to derive the spectrum of $Y(t)$ to be able to analyse its
properties.
This is a simple task, since if the transfer function of the filter
$H(\omega)$ is given, then the spectrum of $Y(t)$, denoted by $S_Y$,
is given by $S_Y(\omega)=|H(\omega)|^2S(\omega)$.
For example, the derivative $X'(t)$ is a Gaussian process
with mean zero and spectrum $S_Y(\omega)=\omega^2S(\omega)$.
The variance of the derivative is $\sigma^2_{X'}=\int
S_Y(\omega)\,d\omega=\lambda_2$.

%The signal analysis for linear filters is important
%for describing the stresses in some part of structure and is particularly
%simple. However, the fatigue process is not dependent on any particular
%form of oscillatory stress but on its total variability and hence the
%fatigue analysis for Gaussian loads is not so simple. An  exception is the
%crossing intensity $\mu(u)$ since the average number of upcrossings of
%the level $u$, $\mu(u)$ is given by the celebrated Rice's formula
%$$
%\mu(u)=f_0\exp(-(u-m)/2\sigma^2).
%$$
%Using spectral moments we have that $\sigma^2=\lambda_0$ while
%$f_0=\frac{1}{2\pi}\sqrt{\frac{\lambda_2}{\lambda_0}}$.

The Gaussian process is a sum of cosine terms with amplitudes defined
by the spectrum; hence, it is not easy to relate the power
spectrum and the fatigue damage. The crossing intensity $\mu(u)$,
which yields the average number of upcrossings of the level $u$, is
given by the celebrated Rice's formula
$$
\mu(u)=f_0\exp(-(u-m)^2/2\sigma^2).
$$
Using spectral moments we have that $\sigma^2=\lambda_0$ while
$f_0=\frac{1}{2\pi}\sqrt{\frac{\lambda_2}{\lambda_0}}$.

Another approach is to model the turning points of
a Gaussian process by a Markov chain, where the so-called Markov matrix is
computed from the specified spectrum $S(\omega)$.
%This analysis
%can be extended for the models that are instantaneous functions of
%Gaussian processes, which further extends the class of models, see Rychlik~\cite{Rychlik88.A1}
%and~\cite{Rychlik96.A1} for review.
Then calculation of rainflow matrices and fatigue damages are
possible. This approach requires a considerable amount of computation,
but often renders accurate results, see Rychlik~\cite{Rychlik88.A1}
and Rychlik et al.~\cite{Rychlik97.A1} for extension to transformed
Gaussian processes.

\subsection*{Fatigue Life Prediction -- Rainflow Method}

In laboratory experiments, one often subjects a specimen of a material
to a constant amplitude load, e.g.
$L(t)= s \sin(\omega t)$ where $s$ and $\omega$ are constants,
and counts the number of cycles (periods) until it breaks.
The number of load cycles $N(s)$ as well as the amplitudes $s$ are
recorded. Note that for small amplitudes, $s<s_{\infty}$,
$N(s)\approx\infty$, i.e.\ no
damage is observed. The amplitude $s_{\infty}$ is called
\emph{the fatigue limit} or \emph{the endurance limit}.
In practice, one often uses a simple model for $N(s)$,
        \begin{equation} %\label{SNmodel}
        N(s)=\left\{ \begin{array}{c@{\quad}l}
        K^{-1} s^{-\beta} & s> s_{\infty},\\
        \infty & s\le s_{\infty},\end{array}\right.
        \end{equation}
where $K$ is a (material dependent) stochastic variable, usually
lognormally distributed, i.e.\ with $K^{-1}=E\gamma^{-1}$ where
$\mbox{ln}(E)\in\mbox{N}(0,\sigma^2_E)$,
and $\gamma$, $\beta$ are fixed constants.


For irregular loads, also called variable amplitude loads, one is often combining the S-N curve with a cycle
counting method by means of
the Palmgren-Miner linear damage accumulation theory, to predict fatigue
failure time. The cycle counting forms equivalent load cycles.
The now commonly used cycle counting
method is rainflow counting and was introduced by Endo~\cite{Matsuishi68.A1} in 1968. It was
designed to catch both slow and rapid variations of the load by
forming cycles by pairing high maxima with low minima even if they are
separated by intermediate extremes.
More precisely, each local maximum is a top of a
hysteresis loop with an amplitude that is computed using rainflow
algorithm. The definition of rainflow cycles as illustrated in
Figure~\ref{FigRFCdef} is due to Rychlik~\cite{Rychlik87.A1}.

\begin{figure}[htbp]
  % Figuren �r ritad i xfig och konverteras med kommandot
  % fig2pstex fig/FigRFCdef_intro
  \begin{center}
    \input{fig/FigRFCdef_intro.pstex_t}
  \end{center}
  \caption{\emph{
      Definition of the rainflow cycle,
      as given by Rychlik~\cite{Rychlik87.A1}.
      From each local maximum $M_k$
      one shall try to reach above the same level, in the backward(left) and
      forward(right) directions, with an as small downward excursion as
      possible. The minimum, of $m_k^-$ and $m_k^+$, which represents the
      smallest deviation from the maximum $M_k$ is defined as the
      corresponding rainflow minimum $m_k^{\rfc}$.
      The $k$:th rainflow cycle is defined as $(m_k^{\rfc},M_k)$.
      }}
  \label{FigRFCdef}
\end{figure}

Let $t_k$ be the time of the $k$:th local maximum and $s_k$
the amplitude of the attached hysteresis loop. Define the total damage by
\begin{equation} \label{Damage}
  D(t)=\sum_{t_k\le t}\frac{1}{N(s_k)}=K\sum_{t_k\le
  t}s_k^\beta=K D_\beta(t)
\end{equation}
where the sum contains all cycles up to time $t$. The
fatigue life time $T^f$, say, is shorter than $t$ if $D(t)>1$.
In other words, $T^f$ is defined as the time when $D(t)$ crosses level
1. A very
simple predictor of $T^f$ is obtained by
replacing $K$ in Eq.~(\ref{Damage}) by a constant, for example the
median value of $K$ equal to $\gamma$.
For high cycle fatigue, the time to failure is long (more than
$10^5/f_0$).  Then for stationary (and ergodic and some other mild
assumptions) loads,
the damage $D_\beta(t)$ can be approximated by its mean
$E[D_\beta(t)]=d_\beta\cdot t$. Here $d_\beta$ is the damage intensity,
i.e.\ how much damage is accumulated per time unit. This leads to
a very simple predictor of fatigue life time
\begin{equation}
\hat T^f=\frac{1}{\gamma d_\beta}.
\end{equation}



\subsection*{Switching Loads -- Rainflow Matrices}

Often the real measurements are gathered in the forms of rainflow
matrices. In the same time there is a need of modelling real loads
that leads to the observed rainflow matrix. In particular the load
can be built up by blocks of stationary load conditions that switch
between each other. The rainflow matrix is then a nonlinear mixture of
the rainflow matrices for the stationary models and for switching between
them. The objective is to model the real loads, see
Johannesson~\cite{Johannesson99.PhD}
for detailed presentation.

When studying switching loads one has to model both the switching
between the subloads and the characteristics of the
different subloads. We will use a hidden Markov
model (HMM) to describe the switching load. This means that the switching
is controlled by a Markov chain, called the regime process, which can
not be observed and therefore is called hidden. Only the switching
load process can be observed,
see e.g.\ Figure~\ref{fig1:SamplePath}. The regime process is defined by
the conditional probabilities of switching
between the different regime states.
This determines the mean length of each subload and the proportion of
the different subloads. The length of a subload is geometrically distributed ($\sim$
exponential).
The subloads are modelled by
min-Max (and Max-min) matrices, see Figure~\ref{fig:TP_Matrix}. This
means that the sequence of local
extremes (also called turning points) are discretized to fixed levels
(often 64 or 128 levels in practice). The transitions from a local
extreme to the next local extreme are approximated by a Markov
chain. This is a 1-step Markov approximation,
as the distribution of the next turning point only depends on the current turning point
and not on the whole history of turning points.
For a thorough description of the models and the algorithms see
Johannesson~\cite{Johannesson99.PhD,Johannesson98.A1}. A summary
without any mathematical details is found in Johannesson
et~al.~\cite{Johannesson97.R2}.

\begin{figure}[ht]
  % Figuren �r ritad i xfig och converteras med kommandot
  % fig2pstex fig/FigTP_Matrix
  \begin{center}
%    \resizebox{!}{60mm}{\input{fig/FigTP_Matrix.pstex_t}}
%    \resizebox{\figwidthA}{!}{\input{fig/FigTP_Matrix.pstex_t}}
    \resizebox{12cm}{!}{\input{fig/FigTP_Matrix.pstex_t}}
  \end{center}
  \caption{\emph{
      Part of a discrete load process where the turning points are
      marked with $\bullet$.
      The scale to the left is the discrete levels.
      The transitions from minimum
      to  maximum and the transitions from maximum to minimum are
      collected in the min-max matrix, $\bm{F}$ and max-min matrix,
      $\bmh{F}$, respectively. The rainflow cycles are
      collected in the rainflow matrix, $\bm{F}^{\rfc}$.
      The figures are the number of observed cycles and the
      grey areas are by definition always zero.
      }}
  \label{fig:TP_Matrix}
\end{figure}
