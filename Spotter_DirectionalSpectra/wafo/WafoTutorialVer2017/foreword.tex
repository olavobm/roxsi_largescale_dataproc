%\svnInfo $Id: foreword.tex 66 2017-09-05 07:01:50Z Georg Lindgren $ 
%$
%
\chapter{Foreword}
\section*{Foreword to 2017 edition}
\addcontentsline{toc}{section}{Foreword to 2017 edition}
\vspace{-0mm}
This \wf{} tutorial 2017 has been successfully tested with \ML{} 2017a on {\sc Windows} 10.

The tutorial for \wf{} 2.5 appeared 2011, with routines tested on \ML{} 2010b. 
Since then, many users have commented on the toolbox, suggesting clarifications 
and corrections to the routines and to the tutorial text.  We are grateful for all suggestions, 
which have helped to keep the \wf{} project alive. 

Major updates and additions have also been made duringing the years, 
many of them caused by new \ML{} versions. 
The new graphics system introduced with  \ML{ 2014b} 
motivated updates to all plotting routines. Syntax changes and warnings for deprecated functions 
have required other updates. 
%The present \wf{} package has been tested on \ML{} up to version 2017a. 

Several additions have also been made. In 2016, a new module, handling non-linear 
Lagrange waves, was introduced. 
A special tutorial for the Lagrange routines is included in the module {\tt lagrange}; \cite{LindgrenPrevosto2017}.  
Two sets of file- and string-utility routines were also added 2016. 

During 2015 the \wf{-}project moved from 
\verb+http://code.google.com/p/wafo/+ to
to 
\verb+https://github.com/wafo-project/+, where it can now be found under the generic 
name \wf{} -- no version number needed. 

In order to facilitate the use of  \wf{} outside the \ML{} environment, 
most of the \wf{} routines have been checked for use with {\sc Octave}. 
On {\tt github}  one can also find a start of a {\sc Python}-version, called {\sf pywafo}. 

Recurring changes in the \ML{} language may continue to cause the command 
window flood with 
warnings for deprecated functions. The routines in this version of \wf{} have 
been updated to work well with \ML{ 2017a}. We will continue to update the toolbox 
in the future, keeping compatibility with older versions.

\section*{Foreword to 2011 edition}
\addcontentsline{toc}{section}{Foreword to 2011 edition}
\vspace{-0mm}
This is a tutorial for how to use the \ML{} toolbox
\wf{} for analysis and simulation of random waves and random fatigue.
The toolbox
consists of a number of \ML{} m-files together with executable
routines from {\sc Fortran} or C++ source, and it requires only a
standard \ML{} setup, with no additional toolboxes.

A main and unique feature of \wf{} is the module of routines
for computation of the exact statistical distributions of
wave and cycle characteristics in a Gaussian
wave or load process. The routines are described in a series
of examples on wave data from sea surface measurements
and other load sequences. There are also sections for
fatigue analysis and for general extreme value analysis.
Although the main applications at hand are from marine and
reliability engineering, the routines are useful
for many other applications of Gaussian and related
stochastic processes.

The routines are based on algorithms for
extreme value and crossing analysis, developed over many years by the
authors as well as many results available in the literature. References are
given to the source of the algorithms whenever it is possible.
These references are given in the \ML{-code} for all the routines and they
are also listed in the Bibliography section of this tutorial.
If the references are not used explicitly in the tutorial;
it means that it is referred to in one of the \ML{} m-files.

Besides the dedicated wave and fatigue analysis routines the toolbox
contains many statistical simulation and estimation
routines for general use, and it can therefore be used as a
toolbox for statistical work.
These routines are listed, but not explicitly explained
in this tutorial.

The present toolbox
represents a considerable development of two earlier toolboxes,
the {\sc Fat} and {\sc Wat} toolboxes, for fatigue and wave analysis, respectively.
These toolboxes were both Version 1; therefore  \wf{} has been
named Version 2. The routines in the tutorial are tested
on \wf-{version} 2.5, which was made available in
beta-version in January 2009 and in a stable version
in February 2011.

The persons that take actively part in creating this tutorial are (in alphabetical order):
{\sl Per Andreas Brodtkorb}\footnote{Norwegian Defense Research Establishment, Horten, Norway.},
{\sl  P\"ar Johannesson}\footnotemark[2], %\addtocounter{footnote}{-1},
{\sl  Georg Lindgren}\footnotemark[3] \addtocounter{footnote}{-2},
{\sl  Igor Rychlik}\footnotemark[4]. \addtocounter{footnote}{1}
\footnotetext[2]{SP Technical Research Institute, Bor{\aa}s, Sweden.}
\footnotetext[3]{Centre for Mathematical Sciences, Lund University, Sweden.}
\footnotetext[4]{Mathematical Sciences, Chalmers, G{\" o}teborg, Sweden.}
Many other people have contributed to our understanding of the problems
dealt with in this text, first of all Professor Ross Leadbetter at
the University of North Carolina at Chapel Hill and
Professor Krzysztof Podg\'orski, Mathematical Statistics, Lund University.
We would also like to particularly thank
Michel Olagnon and Marc Provosto,
at Institut Fran\c{c}ais de Recherches pour l'Exploitation de la Mer (IFREMER),
Brest, who have contributed with many enlightening and fruitful discussions.

Other persons who have put a great deal of effort into \wf{} and
its predecessors FAT and WAT are Mats Frendahl, Sylvie van Iseghem, Finn Lindgren,
Ulla Machado, Jesper Ry\'en, Eva Sj{\" o}, Martin Sk{\" o}ld, Sofia {\AA}berg.

This tutorial was first made available for the beta version of
\wf{} Version 2.5 in November 2009. In the present version some misprints
have been corrected and some more examples added.
All examples in the tutorial have been run with success
on MATLAB up to 2010b.

\cleardoublepage
\chapter*{Technical information}
\addcontentsline{toc}{chapter}{Technical information}
\vspace{-5mm}
\begin{itemize}
\setlength\itemsep{0mm}

\item
\wf{} was released in a stable version in February 2011. 
The most recent stable updated and expanded version of \wf{} can be downloaded from

\verb+https://github.com/wafo-project/+

Older versions can also be downloaded from the \wf{} homepage \cite{WAFO-group2000Wafo}


\noindent
\verb+http://www.maths.lth.se/matstat/wafo/+

\item To get access to the \wf{} toolbox, unzip the downloaded file, identify the {\sf wafo} package and save it in a
folder of your choise. Take a look at the routines \verb+install.m+, \verb+startup.m+, 
 \verb+initwafo.m+ in the \verb+WAFO+ and \verb+WAFO/docs+ folders to learn how \ML{} can find \wf{}. 

\item To let \ML{} start \wf{} automatically, edit \verb+startup.m+ and save it 
in the starting folder for \ML{}.   
\item To start \wf{} manually in \ML{}, add the \verb+WAFO+ 
folder manually to the \ML{-path}
and run \verb+initwafo+. 
\item In this tutorial, the word \verb+WAFO+, when used in path
  specifications, means the full name of the
  \wf{} main catalogue, for instance 
  \verb+C:/wafo/+
  
\item The \ML{} code used for the examples in this tutorial can be found in the
  \wf{} catalogue 
\verb+WAFO/papers/tutorcom/+

The total time to run the examples in fast mode is less than fifteen minutes on a PC from 2017, running 
Windows 10 pro with Intel(R) Core(TM) i7-7700 CPU, 3.6~GHz, 32 GB RAM. All details on execution times 
given in this tutorial relates to that configuration.

\item
\wf{} is built of modules of platform independent \ML{} m-files
and a set of executable files from \verb!C++! and \verb+Fortran+
source files. These executables are platform and \ML{-version} dependent,
and they have been tested with recent \ML{} and {\sc Windows} installations.

\item If you have many \ML{-toolboxes} installed, name-conflicts may occur. 
Solution: arrange the \ML{-path} with \verb+WAFO+ first.

\item
For help on the toolbox, write \verb+help wafo+. 


\item
Comments and suggestions are solicited --- send to 
%
%\noindent
\verb+ wafo@maths.lth.se+

\end{itemize}

%%% Local Variables:
%%% mode: latex
%%% TeX-master: "wafomanual.tex"
%%% End:
